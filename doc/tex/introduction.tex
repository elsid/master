\Introduction

Объектно-ориентированные системы обладают свойством повторяемости конструкций.
Решения множества конкретных задач можно обобщить таким образом,
что структура части системы может быть описана некоторым шаблоном проектирования.
Существуют описания шаблонов, которые рекомендуется использовать при
проектировании таких систем.
Описание шаблона --- структура из классов и связей между ними.
Один из наиболее выразительных языков для описания таких структур ---
UML~\cite{UMLSuperstructure}.
Также существуют анти-шаблоны проектирования.
Некоторые из них можно описать аналогично шаблонам проектирования.

Задача поиска шаблонов проектирования имеет место, её решают различными методами.
Объектом поиска может являться любой программный продукт, который разработан с
использованием парадигмы ООП.
Структуру такой программы или программного комплекса можно представить в виде
UML-диаграммы классов.
Таким образом, задача сводится к поиску одной UML-диаграммы классов в другой.
В этой работе будет предложен метод поиска шаблонов проектирования в
объектно-ориентированных программах.
Основой для модели программы будут UML-диаграммы классов.
Метод позволяет найти все возможные изоморфизмы UML-диаграммы классов шаблону
проектирования.
