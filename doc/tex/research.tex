\chapter{Исследовательский раздел}
\label{cha:research}

\section{Тестирование производительности}

\section{Описание используемых шаблонов проектирования}

\subsubsection{Шаблон abstract factory}

\subsubsection{Шаблон bridge}

\subsubsection{Шаблон visitor}

\section{Поиск шаблонов проектирования в существующих программах}

\subsection{java-design-patterns}

Проект находится в открытом доступе, размещен на ресурсе
github.com~\cite{java-design-patterns}.
Интересен тем, что содержит сборник из 48 примеров реализации различных
шаблонов проектирования на языке \textbf{Java},
что очень хорошо подходит для проверки работы программного комплекса.
Здесь можно увидеть некоторые из особенностей реализации шабонов,
поэксперементировать с разными представлениями шаблона.

\subsubsection{Шаблон abstract factory}

Первый реализованный в проекте шабон --- \textbf{Abstract factory}
(абстрактная фабрика).
Реализация совпадает с разными описаниями.
Шаблон находится, результат представлен в таблице~\ref{table:java-design-patterns-abstract-factory}.

\begin{table}[ht!]
    \centering
    \begin{tabulary}{\textwidth}{|C|C|C|}
        \hline
        Тип элемента & Название целевого элемент & Название элемента шаблона \\
        \hline
        Class & ElfCastle & ConcreteProduct \\
        \hline
        Operation & ElfKingdomFactory::createCastle() & ConcreteFactory::create() \\
        \hline
        Interface & Castle & AbstractProduct \\
        \hline
        Operation & KingdomFactory::createCastle() & AbstractFactory::create() \\
        \hline
        Type & Castle & AbstractProduct \\
        \hline
        Interface & KingdomFactory & AbstractFactory \\
        \hline
        Class & ElfKingdomFactory & ConcreteFactory \\
        \hline
    \end{tabulary}
    \caption{Результат поиска шаблона проектирования abstract factory в примере его реализации}
    \label{table:java-design-patterns-abstract-factory}
\end{table}

\subsubsection{Шаблон bridge}

C шаблоном \textbf{Bridge} (мост) возникла странная ситуация.
В его реализации он не был найден, но был найден в других примерах.
Проблема его реализации в примере заключается в том,
что используется дополнительное обобщение,
чтобы не было дублирования кода.
Класс \textbf{Abstraction} разделен на две части: базовый абстрактный класс
\textbf{MagicWeapon},
который совязан ассоциацией с \textbf{MagicWeaponImp},
являющимся интерфейсом \textbf{Implementor};
и конкретными реализацими: \textbf{BlindingMagicWeapon},
\textbf{FlyingMagicWeapon}, \textbf{SoulEatingMagicWeapon}
(см. рисунок ~\ref{fig:java-design-patterns-bridge}).
Здесь требуется механизм, который позволит описать класс так,
все ассоциации базового класса также являются и ассоциациями производного.
Нужна сущность, которая может объединять иерархии наследования классов в
некоторый надкласс.
UML-диаграммы классов не предоставляют такого механизма.
В данной работе эта проблема осталась не решенной.

\begin{figure}[!ht]
\centering
%\includegraphics{inc/java-design-patterns-bridge.pdf}
\caption{Особенность реализации шаблона bridge}
\label{fig:java-design-patterns-bridge}
\end{figure}

Шаблон был найден в следующих примерах:
\begin{itemize}
\item adapter;
\item decorator;
\item intercepting-filter;
\item mediator;
\item model-view-presenter;
\item null-object;
\item poison-pill;
\item property;
\item service-layer;
\item state;
\item strategy.
\end{itemize}

Для примера \textbf{adapter} резульат представлен в
таблице~\ref{table:java-design-patterns-bridge-in-adapter}.

\subsubsection{Шаблон visitor}
