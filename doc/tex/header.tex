\sloppy

% Настройки стиля ГОСТ 7-32
% Для начала определяем, хотим мы или нет, чтобы рисунки и таблицы нумеровались в пределах раздела, или нам нужна сквозная нумерация.
\EqInChapter % формулы будут нумероваться в пределах раздела
\TableInChapter % таблицы будут нумероваться в пределах раздела
\PicInChapter % рисунки будут нумероваться в пределах раздела

% Добавляем гипертекстовое оглавление в PDF
\usepackage[
bookmarks=true, colorlinks=true, unicode=true,
urlcolor=black,linkcolor=black, anchorcolor=black,
citecolor=black, menucolor=black, filecolor=black,
]{hyperref}

\usepackage[english,russian]{babel}
\usepackage[T2A,T1]{fontenc}
\usepackage[utf8x]{inputenc}
\usepackage{algorithm}
\usepackage{algpseudocode}
\usepackage{amsmath}
\usepackage{amssymb}
\usepackage{array}
\usepackage{bbold}
\usepackage{colortbl}
\usepackage{enumerate}
\usepackage{epstopdf}
\usepackage{fixltx2e}
\usepackage{graphicx}
\usepackage{hyperref}
\usepackage{indentfirst}
\usepackage{listings}
\usepackage{lscape}
\usepackage{minted}
\usepackage{multicol}
\usepackage{multirow}
\usepackage{pdfpages}
\usepackage{pifont}
\usepackage{tabularx}
\usepackage{tabulary}
\usepackage{tikz}
\usepackage{upquote}
\usepackage{verbatim}
\usepackage{xcolor}

\geometry{right=20mm}
\geometry{left=30mm}

\setlength\voffset{-1in}
\setlength\hoffset{-1in}
\setlength\topmargin{0.7874in}
\setlength\oddsidemargin{0.7874in}
\setlength\textheight{9.754932in}
\setlength\textwidth{6.6932993in}
\setlength\footskip{26.148pt}
\setlength\headheight{0cm}
\setlength\headsep{0cm}

\lstset{
  basicstyle=\ttfamily,
  mathescape
}

\usetikzlibrary{arrows,positioning,shadows}

\algnewcommand\algorithmicswitch{\textbf{switch}}
\algdef{SE}[SWITCH]{Switch}{EndSwitch}[1]{\algorithmicswitch\ #1}{\algorithmicend\ \algorithmicswitch}%
\algtext*{EndSwitch}%

\algnewcommand\algorithmiccase{\textbf{case}}
\algdef{SE}[CASE]{Case}{EndCase}[1]{\algorithmiccase\ #1\textbf{:}}{\algorithmicend\ \algorithmiccase}%
\algtext*{EndCase}%

\algnewcommand\algorithmicinput{\textbf{yield}}
\algnewcommand\Yield{\item[\algorithmicyield]}

\MakeRobust{\Call}
