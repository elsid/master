\Referat

В работе предложен метод поиска шаблонов проектирования в объектно-ориентированных
программах на основе алгоритма поиска изоморфных подграфов.
В первой части рассмотрены существующие методы.
Проанализированы алгоритмы поиска изоморфных подграфов.
Во второй части предложена модель представления программы на основе UML-диаграммы классов.
Представление модели в виде графа и алгоритм построения этого графа.
Описан алгоритм поиска изоморфных подграфов в ориентированных графах с множеством
типов дуг.
В третьей части описана реализация программного комплекса, который позволяет
искать шаблоны проектирования в программах, компилирующийся в байт-код
виртуальной машины \textbf{Java}.
В четвертой приведены результаты исследования: поиск шаблонов проектирования
в существующих программах и библиотеках.

Отчет содержит 90 с., 4 ч., 27 рис., 4 табл., 15 источников, 6 приложений.

Ключевые слова: ШАБЛОН ПРОЕКТИРОВАНИЯ, UML-ДИАГРАММА КЛАССОВ, ГРАФ,
ИЗОМОРФИЗМ ПОДГРАФОВ, ОБЪЕКТНО-ОРИЕТИРОВАННЫЕ ПРОГРАММЫ.
